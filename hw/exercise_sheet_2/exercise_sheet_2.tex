\documentclass[a4paper,10pt,twoside,openany]{article}

\usepackage[lang=hebrew]{maths}
\usepackage{hebrewdoc}
\usepackage{stylish}
\usepackage{lipsum}
\let\bs\blacksquare

\setlength{\parindent}{0pt}

%%%%%%%%%%%%
% Styling %
%%%%%%%%%%%%

\usepackage{enumitem}

%%%%%%%%%%%%%
% Counters  %
%%%%%%%%%%%%%

\setcounter{section}{1}     
            
%BIBLIOGRAPHY
\usepackage[
backend=biber,
style=alphabetic,
]{biblatex}
\addbibresource{bibliography.bib} %Imports bibliography file

%%%%%%%%%%
% Title  %
%%%%%%%%%%
\title{
אלגברה ב' - גיליון תרגילי בית 2 \\
סכומים ישרים, מרחבים שמורים, ונילפוטנטיות
\\
\vspace{1cm}
\large{תאריך הגשה: 30.11.2022}
}
\date{}

\begin{document}
\maketitle

\begin{exercise}
יהי
$V$
מרחב וקטורי סוף־מימדי, יהי
$T \in \End_{\mbb{F}}\prs{V}$
ויהיו
$V_1, \ldots, V_k \leq V$
כולם
$T$%
־שמורים וכך שמתקיים
$V = \bigoplus_{i \in [k]} V_i$.

\begin{enumerate}
\item 
הראו כי
\begin{align*}
\ker\prs{T} &= \bigoplus_{i \in [k]} \ker\prs{\left. T \right|_{V_i}} \\
\text{.} \im\prs{T} &= \bigoplus_{i \in [k]} \im\prs{\left. T \right|_{V_i}}
\end{align*}

\item
הראו כי
\begin{align*}
\ker\prs{T - \lambda \id_V} &= \bigoplus_{i \in [k]} \ker\prs{\left. T \right|_{V_i} - \lambda \id_{V_i}}
\end{align*}
לכל
$\lambda \in \mbb{F}$,
והסיקו שהערכים העצמיים של
$T$
הם אלו של כל ה־%
$\left. T \right|_{V_i}$
וגם כי
\begin{align*}
r_{T,a}\prs{\lambda} &= \sum_{i \in \brs{k}} r_{\left. T \right|_{V_i}, a}\prs{\lambda} \\
r_{T,g}\prs{\lambda} &= \sum_{i \in \brs{k}} r_{\left. T \right|_{V_i}, g}\prs{\lambda}
\end{align*}
לכל
$\lambda \in \mbb{F}$
וכאשר
$r_{S,a}\prs{\lambda}, r_{S,g}\prs{\lambda}$
הריבויים האלגברי והגיאומטרי של
$\lambda$
כערך עצמי של
$S \in \End_{\mbb{F}}\prs{V}$.
\end{enumerate}
\end{exercise}

\begin{exercise}
יהי
$T = T_{J_n\prs{0}} \in \Mat_n\prs{\mbb{C}}$
אופרטור הכפל במטריצה
$J_n\prs{0}$
משמאל.
מיצאו בסיס ז'ורדן עבור
$T^2$.
\end{exercise}

\begin{exercise}
תהי
$A = \pmat{-1 & 1 & 0 \\ -2 & 2 & 1 \\ 1 & -1 & -1} \in \Mat_3\prs{\mbb{C}}$
ויהי
$T = T_A$
אופרטור הכפל ב־%
$A$
משמאל.

\begin{enumerate}
\item הראו כי
$T$
נילפוטנטי מאינדקס
$3$
והסיקו כי הוא אופרטור הזזה.

\item
מיצאו בסיס
$B$
של
$\mbb{C}^3$
כך ש־%
$T$
אופרטור הזזה ביחס לבסיס
$B$.
\end{enumerate}
\end{exercise}

\pagebreak

\begin{exercise}
יהי
$V$
מרחב וקטורי סוף־מימדי, יהי
$T \in \End_{\mbb{F}}\prs{V}$
ויהי
$B$
בסיס של
$V$.

\begin{enumerate}
\item נסמן
$A = \brs{T}_B$
ונזכיר כי
\begin{align*}
T_A \colon \mbb{F}^n &\to \mbb{F}^n \\
v &\mapsto Av
\end{align*}
וכי
\begin{align*}
\rho_B \colon V &\to \mbb{F}^n \\
\text{.} \hphantom{lalala} v &\mapsto \brs{v}_B
\end{align*}
הראו כי
\[\text{.} T = \rho_B^{-1} \circ T_A \circ \rho_B\]

\item
הראו כי
$W \leq V$
הינו
$T$%
־שמור אם ורק אם
$\rho_B\prs{W}$
הינו
$T_A$%
־שמור.
\end{enumerate}
\end{exercise}

\begin{exercise}
יהי
$V$
מרחב וקטורי סוף־מימדי, יהי
$T \in \End_{\mbb{F}}\prs{V}$
ויהי
$B = \prs{v_1, \ldots, v_n}$
בסיס של
$V$.
\begin{enumerate}
\item נניח כי
$\brs{T}_B = J_m\prs{\lambda}$.
מיצאו את המרחבים ה־%
$T$%
־שמורים של
$V$.

\item
יהי
$V = \Mat_2\prs{\mbb{C}}$
ויהי
\begin{align*}
T \colon V &\to V \\
\text{.} \pmat{a & b \\ c & d} &\mapsto \pmat{d & a \\ b & 0}
\end{align*}
מיצאו את כל התת־מרחבים ה־%
$T$%
־שמורים של
$V$.
\end{enumerate}
\end{exercise}

\end{document}